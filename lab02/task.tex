\documentclass[a4paper]{article}

\usepackage[russian]{babel}
\usepackage[utf8x]{inputenc}
\usepackage{amssymb,amsthm,amsmath,amscd}
\usepackage{graphicx}

%Оформление теорем, лемм и т.д.
\theoremstyle{plain} % default
\newtheorem{Theorem}{Теорема}
\newtheorem{Lemma}{Лемма}
\newtheorem{Proposition}{Предложение}
\newtheorem{Corollary}{Следствие}
\newtheorem{Statement}{Утверждение}

\title{Многомерный статистический анализ}
\author{Павлов Александр Сергеевич}

\begin{document}

\maketitle

\section*{Лабораторная работа 2.4}

\subsection*{Постановка задачи}

Наблюдается случайная выборка $X = \{x_1, \ldots, x_n\}$ из распределения $\mathcal{N}_N(\mu, \Sigma_0)$, $\Sigma_0$ — заданная матрица. Найти ОМП-оценку параметра $\mu$, выписать распределение оценки. Построить доверительную область для $\mu$, воспользовавшись материалами лекции 1 и 2.

\subsection*{Фрагменты конспекта лекций}

\begin{Theorem}
Пусть $X = (x_1, \ldots, x_n)$, $x_i \sim \mathcal{N}_N(\mu, \Sigma)$. Тогда ОМП-оценками параметров являются:
\begin{equation}
\hat{\mu} = \bar{x} = \frac{1}{n} \sum_{i=1}^n{x_i},
\end{equation}
\begin{equation}
\hat{\Sigma} = \frac{1}{n}A, A = \sum_{\alpha=1}^n{(x_\alpha - \bar{x})(x_\alpha - \bar{x})^T}
\end{equation}
\end{Theorem}

\begin{Theorem}
Пусть $X = (x_1, \ldots, x_n)$, $x_i \sim \mathcal{N}_N(\mu, \Sigma)$ и являются независимыми. Тогда:
\begin{equation}
\bar{x} \sim \mathcal{N}_N(\mu, \frac{1}{n}\Sigma),
\end{equation}
\begin{equation}
A \sim \mathcal{W}_N(\Sigma, n-1),
\end{equation}
$\bar{x}, A$ — независимые
\end{Theorem}

\subsection*{Решение задачи}

Воспользуемся материалами лекций. По теореме 1 \textit{об оценках максимального правдоподобия} получаем оценку для параметра $\mu$ из (1):

\begin{equation*}
\hat{\mu} = \bar{x} = \frac{1}{n} \sum_{i=1}^n{x_i}
\end{equation*}
По теореме 2 \textit{о распределении ОМП-оценок} из (3):
\begin{equation*}
\bar{\mu} \sim \mathcal{N}_N(\mu, \frac{1}{n}\Sigma)
\end{equation*}

\end{document}